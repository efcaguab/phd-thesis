%%%%%%%%%%%%%%%%%%%%%%%%%%%%%%%%%%%%%%%%%
% Classicthesis Typographic Thesis
% LaTeX Template
% Version 1.0 (23/4/12)
%
% This template has been downloaded from:
% http://www.LaTeXTemplates.com
%
% Original author:
% Andr?Miede (http://www.miede.de)
%
% License:
% CC BY-NC-SA 3.0 (http://creativecommons.org/licenses/by-nc-sa/3.0/)
%
% General Tips:
% 1) Make sure to edit the classicthesis-config.file
% 2) New enumeration (A., B., C., etc in small caps): \begin{aenumerate} \end{aenumerate}
% 3) For margin notes: \marginpar or \graffito{}
% 4) Do not use bold fonts in this style, it is designed around them
% 5) Use tables as in the examples
% 6) See classicthesis-preamble.sty for useful commands
%
%%%%%%%%%%%%%%%%%%%%%%%%%%%%%%%%%%%%%%%%%

%----------------------------------------------------------------------------------------
%	PACKAGES AND OTHER DOCUMENT CONFIGURATIONS
%----------------------------------------------------------------------------------------

\documentclass[
		twoside,openright,titlepage,numbers=noenddot,headinclude,%1headlines,
                footinclude=true,cleardoublepage=empty,abstract=on,
                BCOR=5mm,paper=a4,fontsize=11pt, % Binding correction, paper type and font size
                ngerman,american, % Languages
                ]{scrreprt}
%\usepackage{CJK}
%%%%%%%%%%%%%%%%%%%%%%%%%%%%%%%%%%%%%%%%%
% Thesis Configuration File
%
% The main lines to change in this file are in the DOCUMENT VARIABLES
% section, the rest of the file is for advanced configuration.
%
%%%%%%%%%%%%%%%%%%%%%%%%%%%%%%%%%%%%%%%%%

%----------------------------------------------------------------------------------------
%	DOCUMENT VARIABLES
%	Fill in the lines below to enter your information into the thesis template
%	Each of the commands can be cited anywhere in the thesis
%----------------------------------------------------------------------------------------

% Remove drafting to get rid of the '[ Date - classicthesis version 4.0 ]' text at the bottom of every page
\PassOptionsToPackage{eulerchapternumbers, listings,drafting, pdfspacing, subfig,beramono,eulermath,parts}{style/classicthesis}
% Available options: drafting parts nochapters linedheaders eulerchapternumbers beramono eulermath pdfspacing minionprospacing tocaligned dottedtoc manychapters listings floatperchapter subfig
% Adding 'dottedtoc' will make page numbers in the table of contents flushed right with dots leading to them

\newcommand{\myTitle}{The resilience of ecological networks\xspace}
\newcommand{\mySubtitle}{A thesis submitted for the degree of Doctor of Philosophy in Biological Sciences\xspace}
\newcommand{\myDegree}{Doktor-Ingenieur (Dr.-Ing.)\xspace}
\newcommand{\myName}{Fernando Cagua\xspace}
\newcommand{\myProf}{Daniel Stouffer\xspace}
\newcommand{\myOtherProf}{Jason Tylianakys\xspace}
\newcommand{\mySupervisor}{Daniel Stouffer\xspace}
\newcommand{\myFaculty}{College of Science\xspace}
\newcommand{\myDepartment}{School of Biological Sciences\xspace}
\newcommand{\myUni}{University of Canterbury\xspace}
\newcommand{\myLocation}{Christchurch\xspace}
\newcommand{\myTime}{December 2019\xspace}
\newcommand{\myVersion}{version 0.1\xspace}

%----------------------------------------------------------------------------------------
%	USEFUL COMMANDS
%----------------------------------------------------------------------------------------

\newcommand{\ie}{i.\,e.}
\newcommand{\Ie}{I.\,e.}
\newcommand{\eg}{e.\,g.}
\newcommand{\Eg}{E.\,g.}

\newcounter{dummy} % Necessary for correct hyperlinks (to index, bib, etc.)
\providecommand{\mLyX}{L\kern-.1667em\lower.25em\hbox{Y}\kern-.125emX\@}

%----------------------------------------------------------------------------------------
%	PACKAGES
%----------------------------------------------------------------------------------------

\usepackage{lipsum} % Used for inserting dummy 'Lorem ipsum' text into the template

%------------------------------------------------

\PassOptionsToPackage{latin9}{inputenc} % latin9 (ISO-8859-9) = latin1+"Euro sign"
\usepackage{inputenc}

 %------------------------------------------------

%\PassOptionsToPackage{ngerman,american}{babel}  % Change this to your language(s)
% Spanish languages need extra options in order to work with this template
%\PassOptionsToPackage{spanish,es-lcroman}{babel}
\usepackage{babel}

%------------------------------------------------

\PassOptionsToPackage{square,numbers}{natbib}
 \usepackage{natbib}

 %------------------------------------------------

\PassOptionsToPackage{fleqn}{amsmath} % Math environments and more by the AMS
 \usepackage{amsmath}

 %------------------------------------------------

\PassOptionsToPackage{T1}{fontenc} % T2A for cyrillics
\usepackage{fontenc}

%------------------------------------------------

\usepackage{xspace} % To get the spacing after macros right

%------------------------------------------------

\usepackage{mparhack} % To get marginpar right

%------------------------------------------------

\usepackage{fixltx2e} % Fixes some LaTeX stuff

%------------------------------------------------

\PassOptionsToPackage{smaller}{acronym} % Include printonlyused in the first bracket to only show acronyms used in the text
\usepackage{acronym} % nice macros for handling all acronyms in the thesis

%------------------------------------------------

%\renewcommand*{\acsfont}[1]{\textssc{#1}} % For MinionPro
%\renewcommand{\bflabel}[1]{{#1}\hfill} % Fix the list of acronyms

%------------------------------------------------

\PassOptionsToPackage{pdftex}{graphicx}
\usepackage{graphicx}

%----------------------------------------------------------------------------------------
%	FLOATS: TABLES, FIGURES AND CAPTIONS SETUP
%----------------------------------------------------------------------------------------

\usepackage{tabularx} % Better tables
\setlength{\extrarowheight}{3pt} % Increase table row height
\newcommand{\tableheadline}[1]{\multicolumn{1}{c}{\spacedlowsmallcaps{#1}}}
\newcommand{\myfloatalign}{\centering} % To be used with each float for alignment
\usepackage{caption}
\captionsetup{format=hang,font=small}
\usepackage{subfig}

$if(tables)$
\usepackage{longtable,booktabs}
$endif$


%----------------------------------------------------------------------------------------
%	CODE LISTINGS SETUP
%----------------------------------------------------------------------------------------

\usepackage{listings}
%\lstset{emph={trueIndex,root},emphstyle=\color{BlueViolet}}%\underbar} % for special keywords
\lstset{language=[LaTeX]Tex, % Specify the language for listings here
keywordstyle=\color{RoyalBlue}, % Add \bfseries for bold
basicstyle=\small\ttfamily, % Makes listings a smaller font size and a different font
%identifierstyle=\color{NavyBlue}, % Color of text inside brackets
commentstyle=\color{Green}\ttfamily, % Color of comments
stringstyle=\rmfamily, % Font type to use for strings
numbers=left, % Change left to none to remove line numbers
numberstyle=\scriptsize, % Font size of the line numbers
stepnumber=5, % Increment of line numbers
numbersep=8pt, % Distance of line numbers from code listing
showstringspaces=false, % Sets whether spaces in strings should appear underlined
breaklines=true, % Force the code to stay in the confines of the listing box
%frameround=ftff, % Uncomment for rounded frame
frame=single, % Frame border - none/leftline/topline/bottomline/lines/single/shadowbox/L
belowcaptionskip=.75\baselineskip % Space after the "Listing #: Desciption" text and the listing box
}

%----------------------------------------------------------------------------------------
%	HYPERREFERENCES
%----------------------------------------------------------------------------------------

\PassOptionsToPackage{pdftex,hyperfootnotes=false,pdfpagelabels}{hyperref}
\usepackage{hyperref}  % backref linktocpage pagebackref
\pdfcompresslevel=9
\pdfadjustspacing=1

\hypersetup{
% Uncomment the line below to remove all links (to references, figures, tables, etc)
%draft,
colorlinks=true, linktocpage=true, pdfstartpage=3, pdfstartview=FitV,
% Uncomment the line below if you want to have black links (e.g. for printing black and white)
%colorlinks=false, linktocpage=false, pdfborder={0 0 0}, pdfstartpage=3, pdfstartview=FitV,
breaklinks=true, pdfpagemode=UseNone, pageanchor=true, pdfpagemode=UseOutlines,
plainpages=false, bookmarksnumbered, bookmarksopen=true, bookmarksopenlevel=1,
hypertexnames=true, pdfhighlight=/O, urlcolor=webbrown, linkcolor=RoyalBlue, citecolor=webgreen,
%------------------------------------------------
% PDF file meta-information
pdftitle={\myTitle},
pdfauthor={\textcopyright\ \myName, \myUni, \myFaculty},
pdfsubject={},
pdfkeywords={},
pdfcreator={pdfLaTeX},
pdfproducer={LaTeX with hyperref and classicthesis}
%------------------------------------------------
}

%----------------------------------------------------------------------------------------
%	BACKREFERENCES
%----------------------------------------------------------------------------------------

\usepackage{ifthen} % Allows the user of the \ifthenelse command
\newboolean{enable-backrefs} % Variable to enable backrefs in the bibliography
\setboolean{enable-backrefs}{false} % Variable value: true or false

\newcommand{\backrefnotcitedstring}{\relax} % (Not cited.)
\newcommand{\backrefcitedsinglestring}[1]{(Cited on page~#1.)}
\newcommand{\backrefcitedmultistring}[1]{(Cited on pages~#1.)}
\ifthenelse{\boolean{enable-backrefs}} % If backrefs were enabled
{
\PassOptionsToPackage{hyperpageref}{backref}
\usepackage{backref} % to be loaded after hyperref package
\renewcommand{\backreftwosep}{ and~} % separate 2 pages
\renewcommand{\backreflastsep}{, and~} % separate last of longer list
\renewcommand*{\backref}[1]{}  % disable standard
\renewcommand*{\backrefalt}[4]{% detailed backref
\ifcase #1
\backrefnotcitedstring
\or
\backrefcitedsinglestring{#2}
\else
\backrefcitedmultistring{#2}
\fi}
}{\relax}

%----------------------------------------------------------------------------------------
%	AUTOREFERENCES SETUP
%	Redefines how references in text are prefaced for different
%	languages (e.g. "Section 1.2" or "section 1.2")
%----------------------------------------------------------------------------------------

\makeatletter
\@ifpackageloaded{babel}
{
\addto\extrasamerican{
\renewcommand*{\figureautorefname}{Figure}
\renewcommand*{\tableautorefname}{Table}
\renewcommand*{\partautorefname}{Part}
\renewcommand*{\chapterautorefname}{Chapter}
\renewcommand*{\sectionautorefname}{Section}
\renewcommand*{\subsectionautorefname}{Section}
\renewcommand*{\subsubsectionautorefname}{Section}
}
\addto\extrasngerman{
\renewcommand*{\paragraphautorefname}{Absatz}
\renewcommand*{\subparagraphautorefname}{Unterabsatz}
\renewcommand*{\footnoteautorefname}{Fu\"snote}
\renewcommand*{\FancyVerbLineautorefname}{Zeile}
\renewcommand*{\theoremautorefname}{Theorem}
\renewcommand*{\appendixautorefname}{Anhang}
\renewcommand*{\equationautorefname}{Gleichung}
\renewcommand*{\itemautorefname}{Punkt}
}
\providecommand{\subfigureautorefname}{\figureautorefname} % Fix to getting autorefs for subfigures right
}{\relax}
\makeatother

%----------------------------------------------------------------------------------------

\usepackage{style/classicthesis}

%----------------------------------------------------------------------------------------
%	CHANGING TEXT AREA
%----------------------------------------------------------------------------------------

%\linespread{1.05} % a bit more for Palatino
%\areaset[current]{312pt}{761pt} % 686 (factor 2.2) + 33 head + 42 head \the\footskip
%\setlength{\marginparwidth}{7em}%
%\setlength{\marginparsep}{2em}%

%----------------------------------------------------------------------------------------
%	USING DIFFERENT FONTS
%----------------------------------------------------------------------------------------

%\usepackage[oldstylenums]{kpfonts} % oldstyle notextcomp
%\usepackage[osf]{libertine}
%\usepackage{hfoldsty} % Computer Modern with osf
%\usepackage[light,condensed,math]{iwona}
%\renewcommand{\sfdefault}{iwona}
%\usepackage{lmodern} % <-- no osf support :-(
%\usepackage[urw-garamond]{mathdesign} <-- no osf support :-(

\usepackage{color}
\usepackage{fancyvrb}
\newcommand{\VerbBar}{|}
\newcommand{\VERB}{\Verb[commandchars=\\\{\}]}
\DefineVerbatimEnvironment{Highlighting}{Verbatim}{commandchars=\\\{\}}
% Add ',fontsize=\small' for more characters per line

\usepackage{framed}
\definecolor{shadecolor}{RGB}{248,248,248}
\newenvironment{Shaded}{\begin{snugshade}}{\end{snugshade}}
\newcommand{\KeywordTok}[1]{\textcolor[rgb]{0.13,0.29,0.53}{\textbf{{#1}}}}
\newcommand{\DataTypeTok}[1]{\textcolor[rgb]{0.13,0.29,0.53}{{#1}}}
\newcommand{\DecValTok}[1]{\textcolor[rgb]{0.00,0.00,0.81}{{#1}}}
\newcommand{\BaseNTok}[1]{\textcolor[rgb]{0.00,0.00,0.81}{{#1}}}
\newcommand{\FloatTok}[1]{\textcolor[rgb]{0.00,0.00,0.81}{{#1}}}
\newcommand{\ConstantTok}[1]{\textcolor[rgb]{0.00,0.00,0.00}{{#1}}}
\newcommand{\CharTok}[1]{\textcolor[rgb]{0.31,0.60,0.02}{{#1}}}
\newcommand{\SpecialCharTok}[1]{\textcolor[rgb]{0.00,0.00,0.00}{{#1}}}
\newcommand{\StringTok}[1]{\textcolor[rgb]{0.31,0.60,0.02}{{#1}}}
\newcommand{\VerbatimStringTok}[1]{\textcolor[rgb]{0.31,0.60,0.02}{{#1}}}
\newcommand{\SpecialStringTok}[1]{\textcolor[rgb]{0.31,0.60,0.02}{{#1}}}
\newcommand{\ImportTok}[1]{{#1}}
\newcommand{\CommentTok}[1]{\textcolor[rgb]{0.56,0.35,0.01}{\textit{{#1}}}}
\newcommand{\DocumentationTok}[1]{\textcolor[rgb]{0.56,0.35,0.01}{\textbf{\textit{{#1}}}}}
\newcommand{\AnnotationTok}[1]{\textcolor[rgb]{0.56,0.35,0.01}{\textbf{\textit{{#1}}}}}
\newcommand{\CommentVarTok}[1]{\textcolor[rgb]{0.56,0.35,0.01}{\textbf{\textit{{#1}}}}}
\newcommand{\OtherTok}[1]{\textcolor[rgb]{0.56,0.35,0.01}{{#1}}}
\newcommand{\FunctionTok}[1]{\textcolor[rgb]{0.00,0.00,0.00}{{#1}}}
\newcommand{\VariableTok}[1]{\textcolor[rgb]{0.00,0.00,0.00}{{#1}}}
\newcommand{\ControlFlowTok}[1]{\textcolor[rgb]{0.13,0.29,0.53}{\textbf{{#1}}}}
\newcommand{\OperatorTok}[1]{\textcolor[rgb]{0.81,0.36,0.00}{\textbf{{#1}}}}
\newcommand{\BuiltInTok}[1]{{#1}}
\newcommand{\ExtensionTok}[1]{{#1}}
\newcommand{\PreprocessorTok}[1]{\textcolor[rgb]{0.56,0.35,0.01}{\textit{{#1}}}}
\newcommand{\AttributeTok}[1]{\textcolor[rgb]{0.77,0.63,0.00}{{#1}}}
\newcommand{\RegionMarkerTok}[1]{{#1}}
\newcommand{\InformationTok}[1]{\textcolor[rgb]{0.56,0.35,0.01}{\textbf{\textit{{#1}}}}}
\newcommand{\WarningTok}[1]{\textcolor[rgb]{0.56,0.35,0.01}{\textbf{\textit{{#1}}}}}
\newcommand{\AlertTok}[1]{\textcolor[rgb]{0.94,0.16,0.16}{{#1}}}
\newcommand{\ErrorTok}[1]{\textcolor[rgb]{0.64,0.00,0.00}{\textbf{{#1}}}}
\newcommand{\NormalTok}[1]{{#1}}

\usepackage{textcomp} % for backstick
\providecommand{\tightlist}{ % for tight list
  \setlength{\itemsep}{0pt}\setlength{\parskip}{0pt}}

\begin{document}
%\begin{CJK*}{GBK}{song}

\frenchspacing % Reduces space after periods to make text more compact

\raggedbottom % Makes all pages the height of the text on that page

\selectlanguage{american} % Select your default language - e.g. american or ngerman

%\renewcommand*{\bibname}{new name} % Uncomment to change the name of the bibliography
%\setbibpreamble{} % Uncomment to include a preamble to the bibliography - some text before the reference list starts

\pagenumbering{roman} % Roman page numbering prior to the start of the thesis content (i, ii, iii, etc)

\pagestyle{plain} % Suppress headers for the pre-content pages


%----------------------------------------------------------------------------------------
%	PRE-CONTENT THESIS PAGES
%----------------------------------------------------------------------------------------

%*******************************************************
% Little Dirty Titlepage
%*******************************************************
\thispagestyle{empty}
%\pdfbookmark[1]{Titel}{title}
%*******************************************************
\begin{center}
    \spacedlowsmallcaps{\myName} \\ \medskip

    \begingroup
        \color{$maincolor$}\spacedallcaps{\myTitle}
    \endgroup
\end{center}

%*******************************************************
% Titlepage
%*******************************************************
\begin{titlepage}
    %\pdfbookmark[1]{\myTitle}{titlepage}
    % if you want the titlepage to be centered, uncomment and fine-tune the line below (KOMA classes environment)
    \begin{addmargin}[-1cm]{-3cm}
    \begin{center}
        \large

        \hfill

        \vfill

        \begingroup
            \color{$maincolor$}\spacedallcaps{\myTitle} \\ \bigskip
        \endgroup

        \spacedlowsmallcaps{\myName}

        \vfill

        % \includegraphics[width=6cm]{$logo$} \\ \medskip

        \mySubtitle \\ \medskip
        % \myDegree \\
        \myDepartment \\
        \myFaculty \\
        \myUni \\ \bigskip

        \myTime\ -- \myVersion

        \vfill

    \end{center}
  \end{addmargin}
\end{titlepage}

%*******************************************************
% Titleback
%*******************************************************

\thispagestyle{empty}

\hfill

\vfill

\noindent\myName: \textit{\myTitle,} \mySubtitle, %\myDegree,
\textcopyright\ \myTime

\bigskip
%
\noindent\spacedlowsmallcaps{Supervisors}: \\
\myProf \\
\myOtherProf \\
% \mySupervisor
%
\medskip

\noindent\spacedlowsmallcaps{Location}: \\
\myLocation

\medskip

\noindent\spacedlowsmallcaps{Time Frame}: \\
\myTime

%*******************************************************
% Dedication
%*******************************************************
\cleardoublepage

\thispagestyle{empty}
\phantomsection
\pdfbookmark[1]{Dedication}{Dedication}

\vspace*{3cm}

\begin{center}
    \emph{Ohana} means family. \\
    Family means nobody gets left behind, or forgotten. \\ \medskip
    --- Lilo \& Stitch
\end{center}

\medskip

\begin{center}
    Con amor, en memoria de Betty Helena Bermudez. \\ \smallskip
    1964\,--\,2009
\end{center}

%\cleardoublepage\include{FrontBackmatter/Foreword} % Uncomment and create a Foreword.tex to include a foreword

%*******************************************************
% Abstract
%*******************************************************
\cleardoublepage

%\renewcommand{\abstractname}{Abstract}
\pdfbookmark[1]{Abstract}{Abstract}
% \addcontentsline{toc}{chapter}{\tocEntry{Abstract}}
\begingroup
\let\clearpage\relax
\let\cleardoublepage\relax
\let\cleardoublepage\relax

\chapter*{Abstract}
Short summary of the contents in English\dots a great guide by
Kent Beck how to write good abstracts can be found here:


\vfill

\endgroup

\vfill

%\cleardoublepage\include{FrontBackmatter/Publication} % Publications from the thesis page

%*******************************************************
% Declaration
%*******************************************************
\pdfbookmark[0]{Coauthorship declaration}{declaration}
\chapter*{Coauthorship declaration}
\thispagestyle{empty}
Chapter 3 has been extracted from co-authored work.

For Chapter 3, the candidate performed all analysis and wrote the manuscript. All authors contributed to the development of the theoretical framework and edited the text and provided feedback and comments.

On behalf of all co-authors, the undersigned certifys that:

\begin{itemize}
	\item The above statement correctly reflects the nature and extent of the PhD candidate's
    contribution to this co-authored work.
	\item In cases where the candidate was the lead author of the co-authored work he or she
    wrote the text.
	\item Three
\end{itemize}

\bigskip

\noindent\textit{\myLocation, \myTime}

\smallskip

\begin{flushright}
    \begin{tabular}{m{5cm}}
        \\ \hline
        \centering Daniel B. Stouffer \\
    \end{tabular}
\end{flushright}


%*******************************************************
% Acknowledgments
%*******************************************************
\cleardoublepage
\pdfbookmark[1]{Acknowledgments}{acknowledgments}

\begin{flushright}{\slshape
    We have seen that computer programming is an art, \\
    because it applies accumulated knowledge to the world, \\
    because it requires skill and ingenuity, and especially \\
    because it produces objects of beauty.} \\ \medskip
    --- \defcitealias{knuth:1974}{Donald E. Knuth}\citetalias{knuth:1974} \citep{knuth:1974}
\end{flushright}



\bigskip

\begingroup
\let\clearpage\relax
\let\cleardoublepage\relax
\let\cleardoublepage\relax
\chapter*{Acknowledgments}
Thanks to everyone for being awesome

\endgroup

\pagestyle{scrheadings} % Show chapter titles as headings

%\cleardoublepage\include{FrontBackmatter/Contents} % Contents, list of figures/tables/listings and acronyms
% Table of Contents - List of Tables/Figures/Listings and Acronyms

\refstepcounter{dummy}

\pdfbookmark[1]{\contentsname}{tableofcontents} % Bookmark name visible in a PDF viewer

\setcounter{tocdepth}{2} % Depth of sections to include in the table of contents - currently up to subsections

\setcounter{secnumdepth}{3} % Depth of sections to number in the text itself - currently up to subsubsections

\manualmark
\markboth{\spacedlowsmallcaps{\contentsname}}{\spacedlowsmallcaps{\contentsname}}
\tableofcontents
\automark[section]{chapter}
\renewcommand{\chaptermark}[1]{\markboth{\spacedlowsmallcaps{#1}}{\spacedlowsmallcaps{#1}}}
\renewcommand{\sectionmark}[1]{\markright{\thesection\enspace\spacedlowsmallcaps{#1}}}

\clearpage

\begingroup
\let\clearpage\relax
\let\cleardoublepage\relax
\let\cleardoublepage\relax

%----------------------------------------------------------------------------------------
%	List of Figures
%----------------------------------------------------------------------------------------

\refstepcounter{dummy}
%\addcontentsline{toc}{chapter}{\listfigurename} % Uncomment if you would like the list of figures to appear in the table of contents
\pdfbookmark[1]{\listfigurename}{lof} % Bookmark name visible in a PDF viewer

\listoffigures

\vspace*{8ex}
\newpage

%----------------------------------------------------------------------------------------
%	List of Tables
%----------------------------------------------------------------------------------------

\refstepcounter{dummy}
%\addcontentsline{toc}{chapter}{\listtablename} % Uncomment if you would like the list of tables to appear in the table of contents
\pdfbookmark[1]{\listtablename}{lot} % Bookmark name visible in a PDF viewer

\listoftables

\vspace*{8ex}
\newpage

%----------------------------------------------------------------------------------------
%	List of Listings
%----------------------------------------------------------------------------------------

\refstepcounter{dummy}
%\addcontentsline{toc}{chapter}{\lstlistlistingname} % Uncomment if you would like the list of listings to appear in the table of contents
\pdfbookmark[1]{\lstlistlistingname}{lol} % Bookmark name visible in a PDF viewer

\lstlistoflistings

\vspace*{8ex}
\newpage

%----------------------------------------------------------------------------------------
%	Acronyms
%----------------------------------------------------------------------------------------

\refstepcounter{dummy}
%\addcontentsline{toc}{chapter}{Acronyms} % Uncomment if you would like the acronyms to appear in the table of contents
\pdfbookmark[1]{Acronyms}{acronyms} % Bookmark name visible in a PDF viewer

\markboth{\spacedlowsmallcaps{Acronyms}}{\spacedlowsmallcaps{Acronyms}}

\chapter*{Acronyms}

\begin{acronym}[UML]
\acro{DRY}{Don't Repeat Yourself}
\acro{API}{Application Programming Interface}
\acro{UML}{Unified Modeling Language}
\end{acronym}

\endgroup

\cleardoublepage
\pagenumbering{arabic} % Arabic page numbering for thesis content (1, 2, 3, etc)
%\setcounter{page}{90} % Uncomment to manually start the page counter at an arbitrary value (for example if you wish to count the pre-content pages in the page count)

\cleardoublepage % Avoids problems with pdfbookmark

%----------------------------------------------------------------------------------------
%	THESIS CONTENT - CHAPTERS
%----------------------------------------------------------------------------------------

% \ctparttext{You can put some informational part preamble text here. Illo principalmente su nos. Non message \emph{occidental} angloromanic da. Debitas effortio simplificate sia se, auxiliar summarios da que, se avantiate publicationes via. Pan in terra summarios, capital interlingua se que. Al via multo esser specimen, campo responder que da. Le usate medical addresses pro, europa origine sanctificate nos se.} % Text on the Part 1 page describing  the content in Part 1


$body$

% Bibliography

\label{app:bibliography} % Reference the bibliography elsewhere with \autoref{app:bibliography}

\manualmark
\markboth{\spacedlowsmallcaps{\bibname}}{\spacedlowsmallcaps{\bibname}}
\refstepcounter{dummy}

\addtocontents{toc}{\protect\vspace{\beforebibskip}} % Place the bibliography slightly below the rest of the document content in the table of contents
\addcontentsline{toc}{chapter}{\tocEntry{\bibname}}

\bibliographystyle{plainnat}

%\bibliography{bib/bib}
\bibliography{$for(bibliography)$$bibliography$$sep$,$endfor$}

%\end{CJK*}
\end{document}
