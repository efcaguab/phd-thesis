

%------------------------------------------------------------------------------
% 0. Set the encoding of your files. UTF-8 is the only sensible encoding nowadays. If you can't read
% äöüßáéçèê∂åëæƒÏ€ then change the encoding setting in your editor, not the line below. If your editor
% does not support utf8 use another editor!
%------------------------------------------------------------------------------

\PassOptionsToPackage{utf8}{inputenc}
  \usepackage{inputenc}

\PassOptionsToPackage{T1}{fontenc} % T2A for cyrillics
  \usepackage{fontenc}


%------------------------------------------------------------------------------
% 1. Configure classicthesis for your needs here, e.g., remove "drafting" below
% in order to deactivate the time-stamp on the pages
% (see ClassicThesis.pdf for more information):
%------------------------------------------------------------------------------

\PassOptionsToPackage{
  % print version information on the bottom of the pages
  drafting=true,
  % the left column of the toc will be aligned (no indentation)
  tocaligned=false,
  % page numbers in ToC flushed right
  dottedtoc=false,
  % use AMS Euler for chapter font (otherwise Palatino)
  eulerchapternumbers=true,
  % chaper headers will have line above and beneath
  linedheaders=false,
  % numbering per chapter for all floats (i.e., Figure 1.1)
  floatperchapter=true,
  % use awesome Euler fonts for mathematical formulae (only with pdfLaTeX)
  eulermath=false,
  % toggle a nice monospaced font (w/ bold)
  beramono=true,
  % deactivate standard font for loading another one, see the last section at the end of this file for suggestions
  palatino=true,
  style=style/classicthesis-arsclassica % classicthesis, arsclassica
}{style/classicthesis}


%------------------------------------------------------------------------------
% 2. Personal data and user ad-hoc commands (insert your own data here)
%------------------------------------------------------------------------------

\newcommand{\myTitle}{The structure of pollination networks\xspace}
\newcommand{\mySubtitle}{causes \& consequences\xspace}
\newcommand{\myDegree}{Doctoral thesis\xspace}
\newcommand{\myName}{Fernando Cagua\xspace}
\newcommand{\myProf}{Daniel Stouffer\xspace}
\newcommand{\myOtherProf}{Jason Tylianakis\xspace}
\newcommand{\mySupervisor}{Daniel Stouffer\xspace}
\newcommand{\myFaculty}{College of Science\xspace}
\newcommand{\myDepartment}{School of Biological Sciences\xspace}
\newcommand{\myUni}{University of Canterbury\xspace}
\newcommand{\myLocation}{Christchurch, New Zealand\xspace}
\newcommand{\myTime}{December 2019\xspace}
\newcommand{\myVersion}{Version 0.2}

%------------------------------------------------------------------------------
% Setup, finetuning, and useful commands
%------------------------------------------------------------------------------

\providecommand{\mLyX}{L\kern-.1667em\lower.25em\hbox{Y}\kern-.125emX\@}
\newcommand{\ie}{i.\,e.}
\newcommand{\Ie}{I.\,e.}
\newcommand{\eg}{e.\,g.}
\newcommand{\Eg}{E.\,g.}

%------------------------------------------------------------------------------
% 3. Loading some handy packages
%------------------------------------------------------------------------------


% Packages with options that might require adjustments
%------------------------------------------------------------------------------

% change this to your language(s), main language last
% Spanish languages need extra options in order to work with this template
\PassOptionsToPackage{czech,spanish,es-lcroman,main=british}{babel}
\usepackage{babel}

\usepackage{csquotes}
\PassOptionsToPackage{%
  backend=biber,bibencoding=utf8, %instead of bibtex
  %backend=bibtex8,bibencoding=ascii,%
  language=auto,%
  % style=numeric-comp,%
  style=authoryear, % Author 1999, 2010
  %bibstyle=authoryear,dashed=false, % dashed: substitute rep. author with ---
  sorting=nyt, % name, year, title
  maxbibnames=99, % default: 3, et al.
  % giveninits=true
  %backref=true,%
  % natbib=true % natbib compatibility mode (\citep and \citet still work)
}{biblatex}
\usepackage{biblatex}

% math environments and more by the AMS
\PassOptionsToPackage{fleqn}{amsmath}
\usepackage{amsmath}

% General useful packages
%------------------------------------------------------------------------------

\PassOptionsToPackage{pdftex}{graphicx}
\usepackage{graphicx}
\usepackage{scrhack} % fix warnings when using KOMA with listings package
\usepackage{fixltx2e} % Fixes some LaTeX stuff
\usepackage{xspace} % to get the spacing after macros right
\PassOptionsToPackage{printonlyused,smaller}{acronym}
  \usepackage{acronym} % nice macros for handling all acronyms in the thesis
  %\renewcommand{\bflabel}[1]{{#1}\hfill} % fix the list of acronyms --> no longer working
  %\renewcommand*{\acsfont}[1]{\textsc{#1}}
  %\renewcommand*{\aclabelfont}[1]{\acsfont{#1}}
  %\def\bflabel#1{{#1\hfill}}
  \def\bflabel#1{{\acsfont{#1}\hfill}}
  \def\aclabelfont#1{\acsfont{#1}}
%\usepackage{pgfplots} % External TikZ/PGF support (thanks to Andreas Nautsch)
%\usetikzlibrary{external}
%\tikzexternalize[mode=list and make, prefix=ext-tikz/]

% ornaments
\usepackage{adforn}

%------------------------------------------------------------------------------
% 4. Setup floats: tables, (sub)figures, and captions
%------------------------------------------------------------------------------

% better tables
\usepackage{tabularx}
  % increase table row height
  \setlength{\extrarowheight}{3pt}
\newcommand{\tableheadline}[1]{\multicolumn{1}{l}{\spacedlowsmallcaps{#1}}}
% to be used with each float for alignment
\newcommand{\myfloatalign}{\centering}
\usepackage{caption}
\captionsetup{format=hang,font=small}
\usepackage{subfig}

% SO that floats that are larger than this number get their own page
\renewcommand{\floatpagefraction}{.4}

% so can include float barriers if necessary /FloatBarrier
\usepackage{placeins}

% Packages for kable extra
\usepackage{booktabs}
\usepackage{longtable}
\usepackage{array}
\usepackage{multirow}
\usepackage{wrapfig}
\usepackage{float}
\usepackage{colortbl}
\usepackage{pdflscape}
\usepackage{tabu}
\usepackage{threeparttable}
\usepackage{threeparttablex}
\usepackage[normalem]{ulem}
\usepackage{makecell}
% \usepackage{xcolor} - Dont call xcolor because it's already called by classicthesis

% Using czech in babel breaks booktabs
\usepackage{etoolbox}
\preto\tabular{\shorthandoff{-}}


%------------------------------------------------------------------------------
% 5. Setup code listings
%------------------------------------------------------------------------------

% \usepackage{listings}
% % for special keywords
% %\lstset{emph={trueIndex,root},emphstyle=\color{BlueViolet}}%\underbar}
% \lstset{language=[LaTeX]Tex,%C++,
%   morekeywords={PassOptionsToPackage,selectlanguage},
%   keywordstyle=\color{RoyalBlue},%\bfseries,
%   basicstyle=\small\ttfamily,
%   %identifierstyle=\color{NavyBlue},
%   commentstyle=\color{Green}\ttfamily,
%   stringstyle=\rmfamily,
%   numbers=none,%left,%
%   numberstyle=\scriptsize,%\tiny
%   stepnumber=5,
%   numbersep=8pt,
%   showstringspaces=false,
%   breaklines=true,
%   %frameround=ftff,
%   %frame=single,
%   belowcaptionskip=.75\baselineskip
%   %frame=L
% }

%------------------------------------------------------------------------------
% 6. Last calls before the bar closes
%------------------------------------------------------------------------------

% Her Majesty herself
%------------------------------------------------------------------------------
\usepackage{style/classicthesis}
\usepackage{style/classicthesis-arsclassica}

%------------------------------------------------------------------------------
% Fine-tune hyperreferences (hyperref should be called last)
%------------------------------------------------------------------------------

\hypersetup{%
  %draft, % hyperref's draft mode, for printing see below
  colorlinks=true, linktocpage=true, pdfstartpage=3, pdfstartview=FitV,%
  % uncomment the following line if you want to have black links (e.g., for printing)
  %colorlinks=false, linktocpage=false, pdfstartpage=3, pdfstartview=FitV, pdfborder={0 0 0},%
  breaklinks=true, pageanchor=true,%
  pdfpagemode=UseNone, %
  % pdfpagemode=UseOutlines,%
  plainpages=false, bookmarksnumbered, bookmarksopen=true, bookmarksopenlevel=1,%
  hypertexnames=true, pdfhighlight=/O,%nesting=true,%frenchlinks,%
  urlcolor=Maroon, linkcolor=Maroon, citecolor=Maroon, pagecolor=Maroon,%
  % urlcolor=Gray, linkcolor=Gray, citecolor=Gray, pagecolor=Gray,%
  pdftitle={\myTitle},%
  pdfauthor={\textcopyright\ \myName, \myUni, \myFaculty},%
  pdfsubject={},%
  pdfkeywords={},%
  pdfcreator={pdfLaTeX},%
  pdfproducer={LaTeX with hyperref and classicthesis}%
}


%------------------------------------------------------------------------------
% Setup autoreferences (hyperref and babel)
%------------------------------------------------------------------------------
% There are some issues regarding autorefnames
% http://www.tex.ac.uk/cgi-bin/texfaq2html?label=latexwords
% you have to redefine the macros for the
% language you use, e.g., american, ngerman
% (as chosen when loading babel/AtBeginDocument)
%------------------------------------------------------------------------------

\makeatletter
\@ifpackageloaded{babel}%
  {%
    \addto\extrasbritish{%
      \renewcommand*{\figureautorefname}{Figure}%
      \renewcommand*{\tableautorefname}{Table}%
      \renewcommand*{\partautorefname}{Part}%
      \renewcommand*{\chapterautorefname}{Chapter}%
      \renewcommand*{\sectionautorefname}{Section}%
      \renewcommand*{\subsectionautorefname}{Section}%
      \renewcommand*{\subsubsectionautorefname}{Section}%
    }%
    \addto\extrasngerman{%
      \renewcommand*{\paragraphautorefname}{Absatz}%
      \renewcommand*{\subparagraphautorefname}{Unterabsatz}%
      \renewcommand*{\footnoteautorefname}{Fu\"snote}%
      \renewcommand*{\FancyVerbLineautorefname}{Zeile}%
      \renewcommand*{\theoremautorefname}{Theorem}%
      \renewcommand*{\appendixautorefname}{Anhang}%
      \renewcommand*{\equationautorefname}{Gleichung}%
      \renewcommand*{\itemautorefname}{Punkt}%
    }%
      % Fix to getting autorefs for subfigures right (thanks to Belinda Vogt for changing the definition)
      \providecommand{\subfigureautorefname}{\figureautorefname}%
    }{\relax}
\makeatother


% Backreferences

% \usepackage{ifthen} % Allows the user of the \ifthenelse command
% \newboolean{enable-backrefs} % Variable to enable backrefs in the bibliography
% \setboolean{enable-backrefs}{false} % Variable value: true or false
%
% \newcommand{\backrefnotcitedstring}{\relax} % (Not cited.)
% \newcommand{\backrefcitedsinglestring}[1]{(Cited on page~#1.)}
% \newcommand{\backrefcitedmultistring}[1]{(Cited on pages~#1.)}
% \ifthenelse{\boolean{enable-backrefs}} % If backrefs were enabled
% {
% \PassOptionsToPackage{hyperpageref}{backref}
% \usepackage{backref} % to be loaded after hyperref package
% \renewcommand{\backreftwosep}{ and~} % separate 2 pages
% \renewcommand{\backreflastsep}{, and~} % separate last of longer list
% \renewcommand*{\backref}[1]{}  % disable standard
% \renewcommand*{\backrefalt}[4]{% detailed backref
% \ifcase #1
% \backrefnotcitedstring
% \or
% \backrefcitedsinglestring{#2}
% \else
% \backrefcitedmultistring{#2}
% \fi}
% }{\relax}

%------------------------------------------------------------------------------
% Development Stuff
%------------------------------------------------------------------------------
\listfiles
%\PassOptionsToPackage{l2tabu,orthodox,abort}{nag}
%  \usepackage{nag}
%\PassOptionsToPackage{warning, all}{onlyamsmath}
%  \usepackage{onlyamsmath}


%------------------------------------------------------------------------------
% 7. Further adjustments (experimental)
%------------------------------------------------------------------------------

% Changing the text area
%------------------------------------------------------------------------------
%\areaset[current]{312pt}{761pt} % 686 (factor 2.2) + 33 head + 42 head \the\footskip
%\setlength{\marginparwidth}{7em}%
%\setlength{\marginparsep}{2em}%

% Using different fonts
%------------------------------------------------------------------------------

%\usepackage[oldstylenums]{kpfonts} % oldstyle notextcomp
% \usepackage[osf]{libertine}
%\usepackage[light,condensed,math]{iwona}
%\renewcommand{\sfdefault}{iwona}
%\usepackage{lmodern} % <-- no osf support :-(
%\usepackage{cfr-lm} %
%\usepackage[urw-garamond]{mathdesign} <-- no osf support :-(
%\usepackage[default,osfigures]{opensans} % scale=0.95
%\usepackage[sfdefault]{FiraSans}
% \usepackage[opticals,mathlf]{MinionPro} % onlytext

%\usepackage[largesc,osf]{newpxtext}
%\linespread{1.05} % a bit more for Palatino
% Used to fix these:
% https://bitbucket.org/amiede/classicthesis/issues/139/italics-in-pallatino-capitals-chapter
% https://bitbucket.org/amiede/classicthesis/issues/45/problema-testatine-su-classicthesis-style
