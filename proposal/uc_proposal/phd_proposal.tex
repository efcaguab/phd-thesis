\documentclass[a4paper]{report}
\usepackage{afterpage}
\usepackage{hanging}
\usepackage{setspace}
\usepackage{geometry}
\usepackage{todonotes}
\usepackage{doi}
\usepackage[super,sort&compress,comma]{natbib}
\bibliographystyle{unsrtnat}
\usepackage{hyperref}
\hypersetup{
    colorlinks=true,
    citecolor=magenta,
}
\setlength{\parskip}{0.75em}
\onehalfspacing
\pagestyle{plain}


\title{
	{\large Doctoral Thesis Proposal} \\
	{\Huge \textsc{How species interactions shape ecological resilience?}} \\
	{\large Department of Biological Sciences, University of Canterbury}
}
\author{
  {\Large Fernando Cagua}
}
\date{\normalsize \today}

\begin{document}

\maketitle

\chapter*{Summary}
\addcontentsline{toc}{chapter}{Summary}

Natural ecosystems provide important services---like food and water---we humans depend on to a large extent.
Much like the failure of a single key financial institution can trigger unexpected crashes on the stock market, human pressures---such as biological invasions and species extinctions---can cause sudden collapses that severely transform the way ecosystems function.
However, despite its importance, we do not completely understand the dynamics that make ecosystems resilient to collapse.
Because the functioning of ecosystems is largely determined by the network of interactions between the species that inhabit them, my proposed research aims to quantify the role played by species interactions in determining the resilience of ecosystems.
To achieve this, I will focus on networks of mutually beneficial interactions, like those between plants and their pollinators, and use a combination of empirical data, computer simulations and ecological theory.
Ultimately I want to understand why, when and how ecosystem collapses occur, and how to recover from them.

\chapter*{Introduction}
\addcontentsline{toc}{chapter}{Introduction}

From food and freshwater production, to recreation and carbon sequestration, ecosystems provide a wide range of services of considerable value to humans.
Unfortunately, the frequency of undesired ecosystem collapses---like the (often sudden) shift from a transparent to a turbid lake or from a self-sustaining fishery to a collapsed one---is dramatically increasing worldwide\citep{Scheffer2001a}.
When ecosystem resilience is limited, breakdowns are more likely, and their ability to provide those services we depend on is endangered.
Therefore, a necessary step to anticipate, prevent, and reverse ecosystem collapse is to understand the processes that support or undermine ecosystem resilience \citep{Hughes2005, Tylianakis2008}.

Resilience is related to the amount of disturbance that an ecosystem could withstand without collapsing, or, in ecological jargon, undergoing a regime shift---a large, persistent transformation in ecosystem functioning and structure \citep{Holling1973, Gunderson2000}.
Moreover, substantial research indicates that ecosystem functioning and structure are largely determined by the network of interactions formed by species in an ecological community \citep{Bascompte2006, Dobson2006, Tylianakis2008, Reiss2009}.
Since these factors ultimately determine the ecosystem response to disturbances, \textbf{the overall objective of my proposed research is to quantify the role played by species interactions in modulating ecosystem resilience}.

To do so, I will focus on the network of mutualistic interactions between plants and pollinators \citep{Bascompte2006, Bascompte2007, Klein2007}.
These networks, which form the base of pollination systems, play a globally important role in the maintenance of biodiversity and crop production \citep{Bascompte2007, Klein2007}.
Pollination systems are locally important too; for instance two thirds of New Zealand plants are pollinated by birds or insects\citep{Cox2000}, and this includes iconic native plants (like k\={o}whai and p\={o}hutukawa), and economically important crops (like kiwifruit, apples and grapes). However, pollination systems worldwide are currently being disrupted by multiple drivers of human-driven global change \citep{Cox2000}.

Pollination is being affected particularly affected by the simultaneous loss of previously important species, and the introduction of invasive species.
My thesis will concentrate on the resistance and resilience of pollination systems to biotic invasions and defaunation---top components of human-caused global change \todo{ref}.
These two drivers have affected New Zealand with particular intensity, at the same time it remains notably vulnerable \citep{Vitousek1997}.
For instance, 50\% of plant species in New Zealand are introduced \citep{Wilton2000}, and imported social bees are now an important component of pollinator fauna \citep{Lloyd1985, Newstrom2005}.
Moreover, the depletion of native birds \citep{Anderson2003, Robertson2009} is a prime example of how pollination systems in New Zealand are losing key pollinators, plants and habitats\citep{Cox2000}.

In the first chapter of my thesis I will concentrate on biotic invasions.
Specifically I have a twofold objective \textbf{first, I aim to determine which network characteristics shape the its resistance and resilience to invasions}, and second \textbf{to determine how biotic invasions, by affecting existing interactions in the community, reshape network resistance and resilience}.
Empirical, time continuous observations of ecosystem dynamics in networks that have been subjected to invasions are limited.
Therefore, I will use computer simulated communities to estimate the population dynamics of the species in the community, and then I will directly quantify stability properties from fluctuations in the species populations \citep{Bastolla2009, Garcia-Algarra2013}.
Specifically I will simulate communities with different network structures, and analyze which structures are more favorable for the coexistence of the invasive species to co-exist in the community.
I will then in turn, see how the structure itself is changed by the invasive species, and whether the change in structure has stability implications.

However, it has been shown that biotic invasions often occur in ecosystems that have already been degraded by species removal \citep{Bennett2015}.
Understanding how invasions interact with defaunation in ecological networks is a global research priority, and essential for conserving, restoring, and managing New Zealand ecosystems \citep{Newstrom2005}.
In the second chapter, I will follow \textbf{by determining the compound effect of biodiversity loss and invasive species on the resilience of an ecosystem}.
I will be extending the models developed in the first chapter, but focusing on how the loss of species---particularly those with a unique role in the ecosystem---make the ecosystem more or less susceptible to invasions.
Simultaneously, the population models will also serve to determine the the patterns of species extinctions that follows after a successful species invasion.

The first two chapters of my thesis are designed to answer underlying questions of resilience theory, however the underlying aim of my third chapter is to translate the gained insight into useful lessons for ecosystem management.
Ecosystems are complex, non-linear systems that are very difficult to control \todo{ref}.
However, recent work in theoretical physics has highlighted that is indeed possible to regulate them using targeted interventions \citep{Cornelius2013}.
I propose to build upon these findings to \textbf{determine the optimal set of management actions---from both a theoretical and a feasibility perspective---that are required to modify an ecosystem state}.
For that, using a previously collected empirical dataset that characterized the network of interactions before and after an ecological invasion \citep{Bartomeus2008}, I will first find the set of species that can act as drivers of ecosystem state.
Then, using an extension of the population model developed in the first chapter, I aim to determine the characteristics---like the degree of generalization of trophic position---that make an species more likely to serve as a driver of ecosystem state.
Rescuing ecosystems from the brink of collapse and recovering them from undesired shifts is a major goal in conservation science.
Finding a way in which actions targeted to specific species can maximize the ecosystem resilience, will bring us much closer to that goal.

In a world of constant change, building resilience is our best insurance against losing the ecosystem services we value and depend on.
Although we have identified some of the pervasive effects of invasions and defaunation at the species level, we currently do not understand how the changes on ecosystem dynamics is affecting the resilience of the ecosystems as a whole.
The research I propose intends to establish the general theory necessary to answer this question.
Doing so is especially important when the ecosystem response might be inconspicuous until transformation is imminent.
Only by better understanding the dynamics behind species interactions, can we hopefully be better prepared to anticipate, prevent and reverse undesired ecosystem collapses.

\chapter*{Invasibility of pollination systems}
\addcontentsline{toc}{chapter}{Chapter 1}

\subsection*{Why}

Invasions are pervasive.

Invasions can change structure

Structure is important to stability

Which Structures induce invasions.

How invasions change structures/stability

Limited empirical evidence, no generalization, therefore models

Bit models have problems (spiraling coexistence, are wrong. - no actual benefits _ Connect to invasion again)

\subsection*{What}

Novel I will develop a new model without that problem

Novel I will use it and try different structures in initial communities and see effects of apparent competition

See final communities - how structure and therefore stability changes

Are there changes in stable states. Does that depends on the initial or final structure?

\todo[inline]{Invasions are important, problem solving}
\todo[inline]{Start with one of the drivers (invasions)}
\todo[inline]{Identify the gap/problem, solve the gap}

No taking ito account structure

Invaders taking over pollination resources. decreasse effectiveness of pollination for other species

\todo[inline]{Inovation}





\subsection*{How}

In the first chapter of my thesis I will concentrate on biotic invasions.
Specifically I have a twofold objective \textbf{first, I aim to determine which network characteristics shape the its resistance and resilience to invasions}, and second \textbf{to determine how biotic invasions, by affecting existing interactions in the community, reshape network resistance and resilience}.
Empirical, time continuous observations of ecosystem dynamics in networks that have been subjected to invasions are limited.
Therefore, I will use computer simulated communities to estimate the population dynamics of the species in the community, and then I will directly quantify stability properties from fluctuations in the species populations \citep{Bastolla2009, Garcia-Algarra2013}.
Specifically I will simulate communities with different network structures, and analyze which structures are more favorable for the coexistence of the invasive species to co-exist in the community.
I will then in turn, see how the structure itself is changed by the invasive species, and whether the change in structure has stability implications.

Coexistence,

\chapter*{Combined effects of global change}
\addcontentsline{toc}{chapter}{Chapter 2}

Random or top down, down-top, not functional

\chapter*{Controlling ecological networks}
\addcontentsline{toc}{chapter}{Chapter 3}

sdfxc

\chapter*{Research Plan}
\addcontentsline{toc}{chapter}{Research Plan}

asd

\footnotesize
\twocolumn
\bibliography{../../references}
\addcontentsline{toc}{chapter}{References}

\end{document}
