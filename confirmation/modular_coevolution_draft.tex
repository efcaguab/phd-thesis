\documentclass{article}


\usepackage[running]{lineno}
\usepackage[a4paper]{geometry}
\usepackage[draft]{todonotes}
\usepackage{authblk}

\usepackage{graphicx}
\graphicspath{ {figures/} }


\usepackage{graphicx}
\usepackage{epstopdf}
\usepackage{pdflscape}
\usepackage{longtable}
\usepackage{url}

\usepackage{times}

\newenvironment{ecolettcover}{\maketitle}{}

\usepackage[numbers,sort]{natbib}  
\newenvironment{sciabstract}{%
\begin{quote} \bf}
{\end{quote}}
\renewcommand\refname{References}

\title{{\normalsize Attachment 2} \\
	{\LARGE \textsc{The scaling up of coevolution in pollination networks beyond species pairs}}}
\author{Matthew C.\ Hutchinson*, E.\ Fernando Cagua*, and Daniel B.\ Stouffer\\
\normalsize{Centre for Integrative Ecology}\\
\normalsize{School of Biological Sciences}\\
\normalsize{University of Canterbury}\\
\normalsize{Christchurch, New Zealand}}

\date{* Authors share first authorhsip}

\begin{document}
\maketitle
\baselineskip=8.5mm
\vspace{0.4 in}

\begin{itemize}
\item {\bfseries Article Type:} Research Article
\item {\bfseries Running title:} Macroscopic coevolution and pollination
\item {\bfseries Authors' Affiliation:} Centre for Integrative Ecology, School of Biological Sciences, University of Canterbury, Private Bag 4800, Christchurch, New Zealand
\item {\bfseries Corresponding author:} Daniel B.\ Stouffer; daniel.stouffer@canterbury.ac.nz; +64 33642729; School of Biological Sciences, University of Canterbury, Private Bag 4800, Christchurch 8140, New Zealand.
\end{itemize}
\clearpage

\linenumbers

\section*{Abstract}

On a macroevolutionary scale, processes such as coevolution can play a large role in determining who interacts with whom in ecological networks. However, it is unclear whether or how these phenomena are actually detectable in observed assemblages. Here, we aim to bridge this gap by using cophylogenetic analysis to scale up coevolution in 54 empirical pollination networks from the species pair to: \emph{(i)} the entire interaction network and \emph{(ii)} the modular structure of those networks. Our results suggest that the interaction and phylogenetic structures of pollination networks cannot be considered independent with $70\%$ of networks showing significant cophylogeny. These significant patterns of cophylogeny suggest that macroscopic coevolution, as opposed to chance or vicariance, is responsible for the structure of pollination assemblages. At the intermediate scale---modules of tightly interacting groups of species---the patterns expected of macroscopic coevolution are not universally observed. Contrary to previous hypotheses, this suggests that topological modules as we understand them do not represent a fundamental macroscopic unit of coevolution and are more likely determined by multiple eco-evolutionary factors.

\begin{itemize}
\item {\bfseries Keywords:} compartmentalization, diffuse coevolution, ecological networks, guild coevolution, modularity, mutualism, mutualistic networks, cophylogeny
\end{itemize}

\clearpage
\citeindextrue

\section*{1. Introduction}

	Species do not exist in isolation, but are instead constantly interacting with other species in the community. Each of these interactions can impact the fitness of individuals and lead to selection for amplification or avoidance of future interactions \citep{Thompson2005, Futuyma2009}. When two species interact over enough time, reciprocal selection can lead to concurrent evolutionary change in both species \citep{Thompson2005}. This process is generally referred to as coevolution, and it often leads to the evolutionary trajectory of pairs of species, and potentially their descendants, becoming deeply intertwined \citep{Thompson2005}. The coevolution of species can have multiple outcomes, although these most simply can be generalised into two categories: microscopic and macroscopic. Akin to marcoecology---the study of large-scale ecological patterns---macroscopic coevolution can be considered the pattern of coupled divergence of interacting clades that implies a large degree of trait complementarity between interaction partners \citep{Futuyma1983, Brooks1993, Segraves2010}. Conversely, microscopic coevolution---otherwise coadaptation or pairwise coevolution---is the reciprocal trait evolution of interacting species \citep{Janzen1980, Thompson2005}. 
	
	At the microscopic level of species pairs, coevolution has been widely studied \citep{Thompson2005, Futuyma2009}, from corals and their zooxanthellae to symbiotic organelles in eukaryotes. Comparatively, the degree to which macroscopic coevolution is responsible for the structure and functioning of entire communities is poorly understood at the scale of taxonomically rich interacting clades \citep{Thompson2005, Bascompte2014}. Among the ecological systems in which one might expect to observe coevolution at a community scale, the mutualistic assemblages of flowering plants and their pollinators are a logical candidate \citep{Olesen2007, Bascompte2007}. This is because the positive feedback of mutualisms  \citep{Bascompte2007} and myriad examples of extreme pollination-driven coevolution (such as between figs and fig wasps) should make macroscopic coevolutionary signal particularly observable in pollination networks. Furthermore, the notion of diffuse coevolution tenets that, as most species interact with several partners, coevolution should manifest itself above specific pairs of plants and pollinators \citep{Janzen1980, Fox1988, Thompson2005, Nuismer2013}.
	
	Assessing the coevolution of communities is a long-standing question in evolutionary biology \citep{Futuyma1983, Thompson2005, Bascompte2014}, and the ultimate solution requires consideration of both microscopic and macroscopic coevolutionary processes \citep{Thompson2005, Anderson2015}. Nevertheless, important initial insights can be obtained with the quantification of the macroscopic coevolutionary processes observable in the patterns of species interaction and evolutionary history \citep{Futuyma1983, Page2003}. Furthermore, extant interactions---that are typically only possible with trait complementarity \citep{Thompson2005, Segraves2010}---can indicate a certain level of codependent trait evolution at the scale of species' pairs or taxonomically higher \citep{Futuyma1983, Page2003}.  
	
	Cophylogenetic analysis is one common approach that provides insight to the community-wide patterns of interaction and evolutionary history between species---the same patterns expected of macroscopic coevolution \citep{Futuyma1983, Balbuena2013}. At its backbone is the pattern of cospeciation which measures the degree to which two phylogenies match in their speciation and divergence events \citep{Page2003, Clayton2004}. However, this pattern alone can arise from several processes including vicariance, chance, and coevolution \citep{Page2003, Thompson2005, Anderson2015}. Thus, the second aspect of a cophylogenetic pattern that must be present alongside cospeciation is phylogenetic congruence of interactions---that the interactions observed occur between evolutionarily coupled clades \citep{Legendre2002, Balbuena2013}. Cophylogeny is therefore more than just phylogenetic signal of interactions for both interacting groups as it simultaneously considers both phylogenies alongside the extant interactions. 
	
	Though it is a tantalizing possibility, a signal of macroscopic coevolution such as cophylogeny may not extend across an entire pollination network to the same extent that other ecological or evolutionary processes, such as the direct and indirect propagation of disturbance in food webs \citep{Stouffer2011}, traverse interaction networks. Nonetheless, it is widely accepted that diffuse coevolutionary processes may still be manifest in pollination networks at the level of subsets of the community \citep{Thompson2005, Olesen2007, Bascompte2014}. At the same time, it has been observed that closely interacting groups of species, or modules, are a ubiquitous structural feature of ecological communities including pollination networks \citep{Olesen2007, Guimera2010}. Given that coevolution is assumed to act above the level of species pairs \citep{Fox1988, Thompson2005}, it has therefore been hypothesized that modules are an, if not \emph{the}, ecologically relevant product of coevolution \citep{Olesen2007, Bascompte2015}. 

	Despite theory that provides tentative support for this hypothesis \citep{Nuismer2013}, a more directed test of empirical network structure is needed to confirm such results. As with community-wide cophylogeny, multiple constraints must be satisfied for empirically observed modules to show cophylogeny or be considered a product of macroscopic coevolution (Fig.\ \ref{fig:hypotheses}). First, they should reflect the cospeciation of the network such that closely related species tend to be found in the same module. This consistency between modules and phylogenies, however, does not take into account the fact that some interactions, and implied trait complementarity, can be considered more consistent with the cophylogenetic narrative than others when they occur between closely coupled clades \citep{Legendre2002, Page2003, Segraves2010, Balbuena2013}. Therefore secondly, if modules reflect cophylogenetic patterns, they should also be comprised of these phylogenetically congruent interactions, with the less congruent interactions falling between modules.
	
	Here, we study 54 pollination networks from around the world to quantify the degree to which these networks show a significant cophylogenetic pattern. In particular, we search for evidence of cophylogeny at two specific scales: \emph{(i)} network cophylogeny---where any observable cospeciation should be embodied by the interactions of the entire network---and \emph{(ii)} modular cophylogeny---where the modules of a network should tend to contain closely related plant species, closely related pollinator species, and the most congruent interactions between them. Addressing patterns of cophylogeny at these two macroscopic scales represents an important step towards a robust, quantitative integration of coevolutionary processes with the modern theory underpinning community ecology.
	
\section*{2. Material and methods}

\subsection*{(a) Empirical data and phylogeny construction}
We analyzed a dataset comprised of 54 plant-pollinator mutualistic networks from a wide range of locations around the globe and with diverse species assemblages \citep[Table S1]{Rezende2007}. In each of the networks, the presence or absence of interactions is based on observed visitation of flowering plants by their animal pollinators. In total, these networks include 1,388 species of flowering plants, 2,930 species of flower-visiting animals, and over 15,000 interactions.
	Studying the cophylogenetic patterns between two sets of interacting species, such as plants and pollinators, requires an understanding of the evolutionary history of both groups. Central to the accuracy and robustness of our method were rigorously resolved phylogenies of flowering plants and their pollinators, and we followed several steps to generate these. First, to ensure all species identifications were up-to-date, we verified all species' names in the original interaction matrices. Plant names were checked and corrected with the NCBI database (http://www.ncbi.nlm.nih.gov/) whereas we corrected animal names with the gnr\textunderscore resolve function in the R package taxize, which accessed a range of taxonomic databases \citep{TaxizeRef}. Second, we constructed dated phylogenies for both groups with the verified species' names. To do so, we started with a taxonomic tree given by the classification function in taxize with preferential acceptance of classifications returned by the NCBI database \citep{TaxizeRef}. We compiled estimated divergence times of the flowering plants and insect pollinators from accepted phylogenies \citep{Wikstrom2001, Misof2014}. We then used the bladj function from phylocom \citep{PhylocomRef} to obtain branch length estimates for any clades missing from these two studies. Although some of the divergence times we use should be treated as an approximation \citep{Beaulieu2007}, the accurate dating of even a subset of phylogenetic tips, which we have achieved with two accepted phylogenies \citep{Wikstrom2001, Misof2014} can improve the performance of comparative methods such as ours \citep{Webb2000}.

\subsection*{(b) Measuring whole-network cophylogeny}

	To conduct an indirect assessment of macroscopic coevolution in each of our pollination networks, we implemented a recently developed Procrustean method to directly assess cophylogeny in those networks \citep{Balbuena2013}. This approach, referred to as PACo, approaches the cophylogeny problem by optimizing the fit of the phylogeny-interaction graphs of each network \citep{Balbuena2013, Cagua2015}. PACo provides a quantification of the global fit of the phylogenetic objects---a measure of cophylogeny and a proxy for macroscopic coevolution---using the sums of squares residual distance between phylogenetic-interaction graphs \citep{Balbuena2013, Cagua2015}. The smaller the residual distance, the better the fit of the two graphs (\emph{i.e.} phylogenies) and thus a higher degree of cophylogeny reflected in the extant interactions. Similarly, the phylogenetic congruence of each individual interaction is equal to the residual distance between the two corresponding points on the phylogenetic graphs. PACo offers several analysis options including a choice of whether to superimpose the raw phylogenetic graphs (asymmetric) or to normalize both graphs to the same dimensionality before a superimposition is done (symmetric) \citep{Balbuena2013}. The order of superimposition can also be specified (\emph{i.e.} plants on pollinators and vice versa), a decision based on the evolutionary assumptions of the system \citep{Balbuena2013}. In this study, we have focused on the results of the symmetric method where the normalized pollinator graph is superimposed on the normalized plant graph. Nonetheless, our results do not differ qualitatively nor quantitatively when selecting alternative configurations (Fig.\ S1).
	
	We determined the significance of cophylogeny at the network level as captured by PACo by comparing the sum of squared residuals of the Procrustean superimposition of plant and pollinator phylogenies with the same from an ensemble of 1,000 randomizations of the network of interactions between plants and pollinators. In each randomization, we conserved the number of interactions for each species as well as the total number of interactions \citep{Fortuna2010}. A conservative null model such as this does not rely on an evolutionary assumption about which group is driving the cophylogenetic pattern allowing us to test the hypothesis that both the plant and pollinator phylogenetic histories are constrained by the other via their interactions. However, there are several alternate null models (the results for which do not differ qualitatively from the model we present) and selection depends on the evolutionary assumptions made. For example, in host-parasite systems it is perhaps more appropriate to assume that host speciation drives parasite speciation and implement a less constrained null model \citep{Balbuena2013}.
	
\subsection*{(c) Identifying network modules}

	To test whether cophylogeny was consistent with modularity, we first needed to identify modules in the interaction networks---compartments in which species are more likely to interact with each other than the rest of the network \citep{Guimera2010}---given the observed interactions. In order to identify such modules in each of the different networks, we followed the approach proposed by Barber \citep{Barber2007} and implemented in MODULAR \citep{Marquitti2014}, which partitions nodes with a stochastic optimization procedure. While there are several such methods to assess modularity, the method employed here has been demonstrated to perform as well or better than other contemporary  module detection algorithms in bipartite networks \citep{Thebault2013}. 
	
	To determine whether the observed network structure was significantly modular we implemented the same null model used to assess network-wide cophylogeny and compared observed modularity to an ensemble of 1000 randomisations of each network that maintained the degree distribution. This null model has been widely used to assess the significance of modularity in ecological networks \citep{Olesen2007, Fortuna2010}, while at the same time it allows for comparison of our results for network cophylogeny and modularity.

\subsection*{(d) Measuring modular cophylogeny}

The first step we took to quantify the cophylogeny and infer the degree of macroscopic coevolution in modules was to quantify how modules related to the evolutionary histories of both groups of species. To do so, we fitted a likelihood model for discrete character evolution, using a continuous-time Markov model of trait evolution \citep{Fitzjohn2009}. Our assessment compared the degree to which the phylogeny predicts covariance among module assignment of species \citep{Pagel1999}, or in other words, the degree to which closely related species tend to share the same module. 

	We determined the significance of the phylogenetic signal of modules by comparing our results to those that would be obtained after randomly assigning module participants. First, we compared the likelihood that the observed species-module assignment was the product of an evolutionary process versus the corresponding likelihood from 1,000 randomized assignments. Therefore, a significant phylogenetic signal of pollination modules implies that the observed modules are more consistent with this evolutionary process than expected at random.  

	While the degree to which networks as a whole show cophylogeny can be established with PACo \citep{Balbuena2013, Cagua2015}, an assessment of the extent to which cophylogenetic processes characterize topological modules requires subsequent analysis. As noted earlier, the phylogenetic congruence of an interaction is given by the residual of two points in the Procrustean superimposition of the phylogenetic-interaction graphs; therefore, an interaction can be considered more congruent when its contribution to the overall residual of the Procrustean analysis is small \citep{Balbuena2013}. Thus, if modules are the product of cophylogeny, we expected interactions within modules to have a higher degree of congruence (\emph{i.e.} to have smaller residuals on average) than interactions between species in different modules. We tested this assumption using one-factor analysis of variance (ANOVA) of the log-transformed Procrustean partial residuals of each interaction, where the decision to log-transform the data was made to improve the normality but also did not qualitatively affect our results or conclusions.
 	
	We determined the significance of the within-module cophylogeny by comparing our results to those that would be obtained after randomly assigning module participants. We compared the Akaike's Information Criterion (AIC) of the ANOVA that contrasts the degree of cophylogeny within and between observed modules against an ensemble of 1000 AIC values of the same with the randomized module assignments. We randomized the species' module assignments using two approaches. In the first approach, we allowed for an arbitrary number of species in each module and an arbitrary number of modules. In the second, more conservative approach we maintained the observed number of modules and the number of species within each of them. Differences between approaches are not substantial and do not qualitatively affect our results or conclusions, therefore we presented the results of the second, more conservative approach. Results for the first approach can be found in the \emph{Supplementary Material} (Fig. S1).

\section*{3. Results}
	
	 Across our dataset, we found that a large proportion of pollination networks (38 out of 54 networks) could be considered cophylogenetic assemblages (Monte Carlo test, $p<0.05$; Fig.\ \ref{fig:p-values}\emph{A}). The observation of high levels of cophylogeny at a network scale, such as this, therefore provided the necessary baseline from which we could assess how cophylogeny was manifest at the modular scale. Implementing the bipartite modularity optimization of Barber \citep{Barber2007} and a conservative null model (\emph{Materials and Methods}) we observed that $57\%$ (31 out of 54) of networks in our dataset were significantly modular (Monte Carlo test, $p<0.05$; Fig.\ \ref{fig:p-values}\emph{B}). Furthermore, of the significantly modular networks, $74\%$ also showed significant cophylogeny (Fig.\ \ref{fig:metrictileplot}). %(\emph{SI Appendix}, Fig.~).
	
	On the surface, these high levels of cophylogeny and modularity would appear to support the hypothesis that macroscopic coevolution acts at a modular scale. However, as described earlier, this still amounts to circumstantial evidence and a deeper investigation of the relationships between each interaction and the respective phylogenies is required. Specifically, if module participation is the natural result of cophylogeny, we expected that modules based on who tends to interact with whom are consistent with the evolutionary histories of both plants and pollinators---that is, they show cospeciation. Here we instead found that the modules of a network are consistent with the plant and pollinator phylogenies in only $26\%$ and $54\%$ of networks, respectively (Monte Carlo test, $p<0.05$; Figs.\ \ref{fig:p-values}\emph{C}, \ref{fig:p-values}\emph{D}, and \ref{fig:scatter}). However, when considering only the 31 significantly modular networks, the proportion of networks that exhibit phylogenetic signal of module assignment increases to $35\%$ and $58\%$, respectively (Fig.\ \ref{fig:metrictileplot}).%Fig.~ S1).
	
	Our second consideration to assess cophylogeny in modules is the degree of phylogenetic congruence of their constituent interactions. In general, we expected that within-module interactions would tend to be more congruent than between-module interactions if modules represented a cophylogenetic unit (Fig.\ \ref{fig:netcomparison}). We found that observed modules were significantly better than random at explaining interaction congruence in only $26\%$ of the empirical networks (Monte Carlo test, $p<0.05$; Figs.\ \ref{fig:p-values}\emph{E}, \ref{fig:scatter}). Again, this proportion increases slightly to $29\%$ when considering exclusively networks that were significantly modular (Fig.\ \ref{fig:metrictileplot}).%(Fig.~S1).
	
	In total, we have assessed three complementary measures---conservation of the plant phylogeny, conservation of the pollinator phylogeny, and within-module cophylogeny---to quantify whether cophylogeny is manifest in the modules of pollination networks. Across the 54 networks, varying proportions of networks' modules satisfy these cophylogenetic constraints at a modular scale (Fig.\ \ref{fig:p-values} \emph{C-E}). Though each of these were observed at a greater frequency than would be expected at random ($p < 0.001$ in all cases; \citep{Moran2003}) just four networks appear to satisfy constraints for cophylogeny at the modular scale (Fig.\ \ref{fig:scatter}). As such, fulfilling one constraint for cophylogeny does not necessarily facilitate fulfilment of others (Chi-squared tests; both phylogenies conserved, $p<0.001$; plant phylogeny conserved and within-module interaction cophylogeny, $p=0.201$; pollinator phylogeny conserved and within-module interaction cophylogeny, $p=0.002$; all three constraints, $p=0.532$; Figs.\ \ref{fig:p-values}\emph{C-E} and \ref{fig:metrictileplot}). When put together, our results indicate that there is limited evidence at best that modules reflect cophylogenetic patterns or represent a definitive unit of macroscopic coevolution (Figs.\ \ref{fig:metrictileplot} and S1).
	
\section*{4. Discussion}
	
	Coevolution has long been hypothesized to manifest itself in ecological assemblages above the level of species pairs \citep{Janzen1980, Fox1988, Thompson2005, Olesen2007, Bascompte2014}. Here, we address this hypothesis with a formal quantification of cophylogenetic patterns in pollination networks. We have demonstrated that a strong signal is present in pollination networks, with $70\%$ of the networks analyzed exhibiting significant levels of whole-network cophylogeny. Despite current thinking that suggests cophylogeny cannot provide evidence for coevolution \citep{Thompson2005, Anderson2015}, our results in taxonomically rich and globally sourced networks provide a potential counter argument. Many previous studies of cophylogeny address these patterns in specific genera and clades rather than across entire trophic groups such as the flowering plants and pollinators that we focus on here \citep{Nadler1988, Cruaud2012, Nishiguchi1998, Mikheyev2010, Smith2008}. As such, ours is perhaps the first study of cophylogeny in large assemblages of interacting species and the high levels of significantly cophylogenetic networks we observe suggest that this pattern can still arise quite consistently when scaled up. 
	
	Furthermore, for patterns of cophylogeny to be this ubiquitous across such distinct systems, Occam's razor would indicate that it arises from a common process. On the one hand, these patterns can arise due to both biogeographical \citep{Weckstein2004, Thompson2005} and coevolutionary processes \citep{Thompson2005, Godsoe2009}. On the other, multiple previous studies have indicated that phylogenetic congruence in pollination systems is most parsimoniously explained by coevolution \citep{Smith2008, Godsoe2009}. Furthermore, the taxonomically diverse and global nature of our dataset makes it unlikely that vicariance alone can account for the high levels of network cophylogeny we observed here. Likewise, traits are such a strong predictor of interactions in mutualistic networks \citep{Eklof2013, Dehling2014} that strong interaction congruence, like that observed here, is rather unlikely---albeit not impossible---to observe in the complete absence of coupled trait evolution between taxa that maintained trait complementarity across the trees. Our results therefore support the idea that coevolution is indeed an influential process in shaping pollination networks and that many of these communities can be considered macroscopically coevolved assemblages \citep{Thompson2005}. Our results further suggest that an imprint of coevolution can at times be observed in the topological modules of pollination networks. Having said that, it cannot be considered the pervasive structural force that some have suggested \citep{Fox1988, Thompson2005, Olesen2007, Bascompte2014, Bascompte2015}.
		
	The link between traits and interactions in ecological communities \citep{Guimaraes2011, Dehling2014} also offers at least two potential explanations as to why we did not universally see macroscopic coevolution in topological modules. First, in the networks we have studied, closely related pollinators tend to co-occur in modules more than twice as often as closely related plants. This result may be a consequence of convergence in pollination networks via pollination syndromes, whereby plants are thought to converge on traits selected for by pollinator functional groups \citep{Fenster2004}. As functional groups of pollinators are often considered to be broad clades that tend to be more closely related than the plants they visit \citep{Fenster2004}, pollination syndromes may explain why we see such a disparity in terms of phylogenetic conservation of modules. Second, it has been suggested that free-living mutualists tend to converge on traits as opposed to species \citep{Thompson2005}. If traits rather than species converge in pollination networks, future inquiry that incorporates specific trait histories and microscopic coevolutionary dynamics \emph{alongside} phylogenetic and interaction data may perhaps expose a more vivid signal of coevolution \citep{Segraves2010}.
	
	Beyond macroscopic coevolution, modularity in ecological networks is also thought to be influenced by several ecological processes including species' abundance \citep{Krasnov2012}, body mass \citep{Rezende2009} and species richness \citep{Martin2015}. The collective action of these processes and others may act to effectively layer over the products of coevolution, resulting in the weak signal we tend to observe in network modules. Our results also show variation in the extent to which coevolution within modules is quantifiable---for example, the pollinator phylogeny is conserved by the modular structure in roughly half the networks we assess---suggesting that the contribution of ecological processes to modular structure may differ on both spatial and temporal scales \citep{Bascompte2014}. Therefore, attempts to tease apart the contributions of ecological, evolutionary, and coevolutionary processes to the modular structure of ecological networks may be as important as attempts that aim to tease apart the ecological mechanisms alone \citep{Vazquez2009, Kaiser2014}.
	
	Conversely, it is also possible that a modular structure will always fail to serve as a definitive unit of macroscopic coevolution in pollination networks when examined at the whole-network scale. For example, it is entirely plausible, and perhaps even likely, that the role of macroscopic coevolution in contributing to module structure differs between the modules observed in a network. Variation in the strength of macroscopic coevolution observable at a module scale may arise from differences in the period of shared evolutionary history of participant species \citep{Bascompte2015}, infiltration of modules by exotic species \citep{Traveset2013}, or merely the aforementioned suite of ecological determinants. Furthermore, the balance of pollinator species to plant species in a module may also indicate the extent to which coevolution has shaped the assemblage, with suggestions that balanced, or symmetrical, modules are more likely the product of coevolution and predisposed to reciprocal selection whereas asymmetrical modules are more likely driven by abundance \citep{Hagen2012, Rasmussen2013}. Thus, a description of coevolution at a modular scale in ecological networks requires not only a consideration of the multiple determinants of ecological modularity, but also an explicit focus on individual modules and their constituent species. \\


\noindent{\bf Data accessibility:} All interaction networks can be found on the Web of Life database (http://www.web-of-life.es). Summaries of the datasets used in this article can also be found as part of the supplementary material. Both plant and pollinator phylogenies for each interaction network are available through GitHub: http://github.com/stoufferlab/to-be-added-upon-publication/ \\

\noindent{\bf Authors' contributions:} 
M.\ C.\ H.\ and E.\ F.\ C.\ contributed equally to the work; M.\ C.\ H.\ and D.\ B.\ S.\ designed research; M.\ C.\ H.\ and E.\ F.\ C.\ performed research; E.\ F.\ C.\ analyzed data; M.\ C.\ H.\ led the writing; and E.\ F.\ C.\ and D.\ B.\ S.\ contributed substantially to the writing.\\

\noindent{\bf Competing interests:} We declare no competing interests.\\

\noindent{\bf Acknowledgements:} The authors thank all those without whose data this work would be impossible. We thank T.\ Poisot, J.\ M.\ Olesen, S.\ L.\ Nuismer, B.\ B.\ Mora, M.\ P.\ Gaiarsa, and S.\ Saavedra for invaluable comments on an early manuscript. We also acknowledge T.\ Poisot for his role in developing the software integral to this study.\\

\noindent{\bf Funding:} M.\ C.\ H.\ acknowledges the support of the University of Canterbury Summer Scholarship program. E.\ F.\ C.\ acknowledges the support of both a University of Canterbury Doctoral Scholarship and a Rutherford Discovery Fellowship (to D.\ B.\ S.\ ). D.\ B.\ S.\ acknowledges the support of a Rutherford Discovery Fellowship, administered by the Royal Society of New Zealand.\\

\clearpage


\begin{figure}[ht]
\centerline{\includegraphics*[width=1.00\textwidth]{fig_hypotheses}}
\caption{Graphical representation of our key hypotheses for modular cophylogeny in pollination networks. (A) We show a significantly modular network composed of three modules of tightly interacting species as indicated by the different colors (e.g., the green module containing species u, v, O, and P). This matrix format is one way to represent a binary bipartite ecological network; the pollinator species are on the x-axis (O--T), the plants on the y-axis (u--z), and a solid square at their intersection indicates the presence of an interaction between those species. (B) The modules in this network are highly cophylogenetic since they are made up of closely related species in both phylogenies and the evolutionary histories of these interacting species mirror each other strongly. (C and D), Despite the fact that this network is significantly modular, there is no evidence for modular cophylogeny since there is no tendency of its three modules to be composed of closely related species nor for species in extant interactions to exhibit comparable evolutionary histories. In both (A) and (C), the shading of the interactions is indicative of that interactions' cophylogenetic signal where the darker the shade the higher the signal (\emph{Materials and Methods}).}
\label{fig:hypotheses}
\end{figure}
\clearpage

\begin{figure}[ht]
\centerline{\includegraphics*[width=0.57\textwidth]{fig_netresults}}
\caption{The structure of empirical pollination networks provides varying degrees of support for the different constraints for whole-network and modular cophylogeny. (A)-(E) We show the probability densities of the p-values associated with our hypothesis testing of each criteria: (A) network cophylogeny, (B) network modularity, (C) phylogenetic signal of pollinator modules, (D) phylogenetic signal of plant modules, and (E) within-module cophylogeny.
In each panel, we indicate the the proportion of networks that are significant for each measure compared to the corresponding null hypothesis (\emph{Materials and Methods}). The red circle, blue square, and green diamond on each plot show the values for three representative networks to indicate the variability within as well as across individual networks.}
\label{fig:p-values}
\end{figure}
\clearpage

\begin{figure}[ht]
\centerline{\includegraphics*[width=1.1\textwidth]{fig_presentedresults}}
\caption{Graphical summary of the results of our analysis of cophylogeny in empirical pollination networks. Each column is for a different empirical network and the rows show the outcome of different statistical tests. In all rows a dark blue cell indicates a significant result for the network whereas light blue indicates the result was non-significant. Row 1 shows the proportion of significantly modular networks. Row 2 shows the proportion of significantly cophylogenetic networks. Rows 3 \& 4 show the proportion of networks with phylogenetic signal of the pollinator and plant modules, respectively. Row 5 shows the support for within-module cophylogeny of the networks.}
\label{fig:metrictileplot}
\end{figure}
\clearpage

\begin{figure}[ht]
\centerline{\includegraphics*[width=0.90\textwidth]{fig_scatter}}
\caption{Few empirical pollination networks simultaneously support the hypothesis of substantial modular cophylogeny. For each of the 54 empirical networks, we plot the phylogenetic signal of pollinator modules (on the x-axis) versus the phylogenetic signal of the plant modules (on the y-axis). Networks to the left of the vertical dashed line and below the horizontal dashed line exhibit significant phylogenetic signal for pollinators and plants, respectively. The color of symbol indicates whether the networks' modular structure significantly explained the variation of interaction-level cophylogeny; dark blue indicates those networks whose modules have a significant tendency to contain the most congruent interactions whereas light blue indicates the opposite. Only four of the empirical networks fulfil all three constraints.}
\label{fig:scatter}
\end{figure}
\clearpage

\begin{figure}[ht]
\centerline{\includegraphics*[width=1.00\textwidth]{fig_compare}}
\caption{There is substantial variation in the extent to which the modular structure of a network captures macroscopic cophylogeny. We show here how two representative networks fit the constraint that within-module interactions should be more congruent than between-module interactions. The color of the within- and between-module interactions in each network is given by the mean cophylogentic signal of all interactions in that group. (A) The constraint for within-module cophylogeny is fulfilled by Network 2 since the interactions within modules tend to show significantly more cophylogeny than those falling between modules. (B) In contrast, the constraint for within-module cophylogeny is not fulfilled by Network 12 since within-module interactions tend to show less cophylogeny than the between-module counterparts. In cases like this---which are the norm of our study---the whole network provides stronger support for cophylogeny than do topological modules.}
\label{fig:netcomparison}
\end{figure}
\clearpage

\bibliographystyle{prsb.bst}
\bibliography{modcoevobib}
\clearpage

%end of document!
\end{document}
